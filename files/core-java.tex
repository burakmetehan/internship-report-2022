\section{Core \texttt{Java}}

\subsection{Environment Setup and First Program}

Since we have been using \texttt{\texttt{Java}} during our internship, our mentor introduced necessary development environments and we installed the \texttt{JAVA SDK} and \texttt{IntelliJ IDEA}.

After we discussed what software is and how it works on a computer and programming languages, we talked about \texttt{Java}'s history and working structure. Then, we continued by creating the first project on \texttt{IntelliJ IDEA}, and we started learning the core \texttt{Java}.


\subsection{Introduction to \texttt{Java}}

We started with the basics by the standard learning curve of a programming language. The first week's four days were about \texttt{Java}'s basics and core concepts. Since syntax and basic concepts of \texttt{\texttt{Java}} are pretty similar to \texttt{C} and \texttt{C++}, understanding the topics was not that hard for me at the beginning of the week.

While learning the concepts and basics, we usually practiced them. The usual process was that after we first coded the exercise, we examined the sample solutions other team members coded. This approach has been constructive for me at times because reviewing other people's codes helps the learning phase.

\subsubsection{Basics of \texttt{Java}}

While learning the basics, the part that was remarkable to me was the Strings because there are some classes, such as \texttt{StringBuffer} and \texttt{StringBuilder}, to construct Strings. We also talked about advanced \texttt{Java IO} and the concept of `\texttt{serialization}', the conversion of the state of an object into a byte stream, which was new to me.

After these basics, some advanced features such as enumerations, interfaces, abstract, generic classes, and wildcards, which are new to me, are discussed.

\subsubsection{Collections and OOP}

I knew some data structures from the courses I have taken. Many of these data structures are implemented in Java and called collections. We discussed the collection framework hierarchy of \texttt{Java} and learned these collections. Also, we examined the working ways of \texttt{hashCode()} and \texttt{equals()}, important methods for these collections to work properly.

Afterward, we covered the basics, such as scopes, constructors, access modifiers, setter and getter methods, method overloading, and principles, such as inheritance, abstraction, encapsulation, and polymorphism, of Object Oriented Programming (OOP). Since I was familiar with concepts from CENG 242: Programming Language Concepts, I did not have a hard time with these concepts.

\subsubsection{JavaBeans and JDBC}

The last day of the week, when we got out of the basics of \texttt{Java} and learned databases and database connections by using \texttt{Java}, was a day that I learned new information and was challenged for the first time.

We started by talking about the \texttt{JavaBeans}. By learning and reading about \texttt{JavaBeans}, I realized that many things are standardized in \texttt{Java}, and implementations are competed instead of the standards. Additionally, we talked about Maven, a build automation tool, and installed it in addition to MySQL, which was going to be used as a database. After the installations, we dived into the \texttt{Java} Database Connectivity (JDBC),  \texttt{Java} API that mainly manages to connect to a database, and sending SQL statements to databases.


\subsection{Git \& Bitbucket}

To our use during the internship, we were provided a Bitbucket account. We were asked to use Git and \texttt{push} our codes to Bitbucket. Since we were asked to use Git, the basics and primary usage of Git and Bitbucket were shown. 
