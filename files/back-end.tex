\subsection{Back-end}

As mentioned in the requirements, I used \texttt{Spring} for the back-end. The main folder structure can be seen \hyperref[back-end-tree]{at the appendicies}. \texttt{Spring Security} is active, and \texttt{JWT} tokens, which need to be inside the request's header, are used for authentication. After security is passed, the request is handled by controllers.

In the back-end part, there are nine main folders:
\begin{itemize}
  \item \texttt{config}: contains the configurations and data loader.
  \item \texttt{controller}: contains request (Rest) controllers. If the requests are authorized, they are met by the related controller. When a request is met, necessary business logic is run.
  \item \texttt{entity}: contains entity classes. These entity classes are used in the tables of the database.
  \item \texttt{exception}: contains the user-defined exceptions and global exception handler.
  \item \texttt{filter}: contains filter classes. All requests go through the filters.
  \item \texttt{model}: contains model classes. These classes are used for data transfer between client, server, and database.
  \item \texttt{repo}: contains repositories. These interfaces are used to access the database by using \texttt{Spring JPA}.
  \item \texttt{service}: contains services. These classes consist of the business layer functions.
  \item \texttt{util}: contains utilities.
\end{itemize}
These are the main folders consisting of several files; I will explain them in detail.


\subsubsection{\texttt{config}}

The classes inside this folder are used for the configuration of \texttt{Spring} and data loading when \texttt{Spring} runs the app. After the app starts running, the data loader is not used, although configurations influence the app's answer.

\begin{figure}[ht]
  \label{back-end-config-tree}
  \centering
  \begin{forest}
    pic dir tree,
    where level=0{}{% folder icons by default; override using file for file icons
      directory,
    },
    [config
      [AuthEntryPoint.java, file]
      [DataLoader.java, file]
      [PasswordConfig.java, file]
      [WebSecurityConfig.java, file]
    ]
  \end{forest}
  \caption{Structure of config folder.}
\end{figure}

\paragraph{\texttt{AuthEntryPoint.java:}} When authorization is failed in \texttt{Spring Security}, the function inside this class is called and returns a response with the \texttt{UNAUTHORIZED} status code.

\paragraph{\texttt{PasswordConfig.java:}} This class indicates the password encoder which is used through the app.

\paragraph{\texttt{WebSecurityConfig.java:}} It is configuration of \texttt{Spring Web Security} that checks all the requests for authorization and roles. Also, if an exception occurs, it handles the exception with the help of a global exception handler. 

The web security is a little bit confusing and not easy to handle. Especially, \texttt{JWT} token-based authentication challenged me. I spent almost one day understanding what \texttt{JWT} token and \texttt{JWT} token-based authentication are. By reading documents and examining examples, I implemented \texttt{JWT} token-based authentication for my security.

\paragraph{\texttt{DataLoader.java:}} This class is run when the application is run. It checks the database for the default admin user and roles, called \texttt{ROLE\_ADMIN} and \texttt{ROLE\_USER}. If the database is missing, it creates admin user and roles. After learning the \texttt{ApplicationRunner} interface, it was easy to implement.


\subsubsection{\texttt{controller}}

Controllers are the essential elements of back-end development. Controller files are used with \texttt{@RestController} annotation, which indicates that the class is a Rest API controller. Each class has \texttt{@RequestMapping} annotation that shows the path of the controller; that is, the requests coming to the specified path are handled by that class.

The classes include several mappings for different request types such as \texttt{GET}, \texttt{POST}, \texttt{PUT}, or \texttt{DELETE}. When a request is sent to the back-end, \texttt{Spring} sends it to a suitable controller according to \texttt{@RequestMapping}. When a request arrives at the class, it handles it according to its type.

Since controllers are not tricky, I could easily code and map the requests. The most challenging part of the controllers was the authentication of functions. Since some operations, such as deleting users or books, cannot be done by regular users but admins, I needed to secure the related functions. After researching, I managed to block the regular user using some operations by using \texttt{@Secured} annotation. This annotation takes a string which is the privileged role, in my case \texttt{ROLE\_ADMIN}. When this annotation is used, \texttt{Spring Security} checks for the additional role.

\begin{figure}[ht]
  \label{back-end-controller-tree}
  \centering
  \begin{forest}
    pic dir tree,
    where level=0{}{% folder icons by default; override using file for file icons
      directory,
    },
    [controller
      [AuthController.java, file]
      [BookController.java, file]
      [BookListController.java, file]
      [UserController.java, file]
    ]
  \end{forest}
  \caption{Structure of controller folder.}
\end{figure}

\paragraph{\texttt{AuthController.java:}} This class handles the requests coming to the path `\texttt{/auth}' and has two \texttt{POST} mapping.

This class is responsible for authentication operations. It can be used to check the \texttt{JWT} token's validity or log in. Using the information inside the request body, I checked the necessary information and sent the response to the client. The responses include some required information for the front-end, such as validation information, admin status, or \texttt{JWT} token.

\paragraph{\texttt{BookController.java:}} This class handles the requests coming to the path `\texttt{/books}'. It has six \texttt{GET}, one \texttt{POST}, one \texttt{PUT}, and one \texttt{DELETE} mapping. 

This class handles book operations such as searching, adding, updating, or deleting. I need a particular concern in this class: Pageable and List data. Since I needed pageable and list data in the front-end while searching the book, I coded two variances of each search function. This class can handle both id and name searches. In name searches, there is no need to provide the full name of the book. Instead, providing a letter or word inside the book name is enough.

\paragraph{\texttt{BookListController.java:}} This class handles the requests coming to the path `\texttt{/}' and has two \texttt{PUT}. Although the main path is \texttt{/}, the two functions inside this class have special path mapping for its \texttt{PUT} mapping. This class is responsible for the favorite and read list operations. 

According to the traditional way, updating something is done with \texttt{PUT} request. Since adding or removing a book from read or favorite lists is updating the list, I decided to use \texttt{PUT} requests. However, there were four operations: adding or removing from the read list and adding or removing from the favorite list. Firstly, I divided all operations into four functions and realized it was not a good idea because path mapping was not nice. 

So, a problem came up about how to handle these operations. I decided to have two main functions and paths for read and favorite lists. In main paths \texttt{/read} and \texttt{/fav}, I handled read and favorite list operations, respectively. A request parameter is needed as well as the book and user ids to acquire the necessary information, which book will be added or removed.

\paragraph{\texttt{UserController.java:}} This class handles the requests coming to the path `\texttt{/users}'. It has six \texttt{GET}, one \texttt{POST}, one \texttt{PUT}, and one \texttt{DELETE} mapping.

This class is similar to the book controller and is responsible for user operations such as searching, adding, updating, or deleting. Like in the book controller, I also need a particular concern, Pageable and List data, and I solve this problem the same way in the book controller.


\subsubsection{\texttt{entity}}

Entities were my main data classes. Thanks to Hibernate, I was also able to create the database tables by using \texttt{@Table} annotations, so the tables were created automatically. Around the back-end, such as between data and business layers, I used these entity classes; however, I used models while sharing and transmitting information between the client and server.

The main problem with entity classes was the mutual data types. A \texttt{User} includes \texttt{Set<Book>} inside it, and A \texttt{Book} includes \texttt{Set<User>} inside it. This was a problem when this information was sent to the client side due to this mutuality. For example, when a \texttt{User} is sent to the client, its read list is also sent. However, inside the read list, there are books that contain the \texttt{User} that is being sent to the client. Therefore, there was an infinite loop while parsing the data. I solved it by using \texttt{@ManyToMany}, and \texttt{@JsonManagedReference} annotations.

\begin{figure}[ht]
  \label{back-end-entity-tree}
  \centering
  \begin{forest}
    pic dir tree,
    where level=0{}{% folder icons by default; override using file for file icons
      directory,
    },
    [entity
      [Book.java, file]
      [EntityBase.java, file]
      [Role.java, file]
      [User.java, file]
    ]
  \end{forest}
  \caption{Structure of entity folder.}
\end{figure}

\paragraph{\texttt{Book.java:}} This is an entity class that extends \texttt{EntityBase} for a \texttt{Book} and is responsible for holding information of a book. It has `name', `author', `page count', `type', `publisher', and `publication date' fields as well as the data fields in \texttt{EntityBase}. Also, it has many-to-many relations with \texttt{User} class. 

\paragraph{\texttt{User.java:}} This is an entity class that extends \texttt{EntityBase} for a \texttt{User} and is responsible for holding information of an user. It has `username', `password', `read list', `favorite list', and `roles' fields as well as the data fields in \texttt{EntityBase}. Also, it has many-to-many relations with \texttt{Book} and \texttt{Role} classes. 

\paragraph{\texttt{Role.java:}} This is an entity class that extends \texttt{EntityBase} for a \texttt{User} and is responsible for holding information of a role. It has the `name' field and the data fields in \texttt{EntityBase}. Also, it has many-to-many relations with \texttt{User} class. 

\paragraph{\texttt{EntityBase.java:}} This is an entity class that is the base of other entities. This holds the general information such as `id', `creation date', `update date', or `activity'.


\subsubsection{\texttt{exception}}

Exceptions are used at several points to provide information to the exception handler. The exception handler catches the exceptions and sends a response to the client. The primary purpose of the exception handler is security because default \texttt{Spring} errors exploit some system information. Therefore, I used special exceptions and an exception handler to provide necessary but unimportant information outside.

Creating new exception classes was not hard. However, adjusting the exception handler is a little tough because the order of the functions can be a problem. I solved this problem using \texttt{@Order} annotations and attention.

\begin{figure}[ht]
  \label{back-end-exception-tree}
  \centering
  \begin{forest}
    pic dir tree,
    where level=0{}{% folder icons by default; override using file for file icons
      directory,
    },
    [exception
      [BadRequestException.java, file]
      [BookListException.java, file]
      [BookNotFoundException.java, file]
      [ConflictException.java, file]
      [GlobalExceptionHandler.java, file]
      [RoleNotFoundException.java, file]
      [UserNotFoundException.java, file]
    ]
  \end{forest}
  \caption{Structure of exception folder.}
\end{figure}

The names of the files explain what the classes are for and what they do. These are used for specified exceptions in related positions.


\subsubsection{\texttt{filter}}

My app contains only one filter, which is checking the \texttt{JWT} token. When a user or admin is logged in to the system, I send a new \texttt{JWT} token. The reason for that is transmitting username and password in all requests. Instead, I use a \texttt{JWT} token for authentication. All requests go through the filter for \texttt{JWT} token check. If the \texttt{JWT} token does not exist in the header or it is expired, \texttt{BadRequestException} is thrown, and a response is returned with \texttt{BAD\_REQUEST} status code.

This part challenged me because I did not know \texttt{JWT} and how to check it. For this filter and the \texttt{JWT} authorization, I spent almost one day. However, the main problem was not that the subject was difficult but that I misunderstood some concepts and topics and needed to change the security configuration quite a bit. By getting help from the internet and reading, I solved the problem in security configuration and filter.

\begin{figure}[ht]
  \label{back-end-filter-tree}
  \centering
  \begin{forest}
    pic dir tree,
    where level=0{}{% folder icons by default; override using file for file icons
      directory,
    },
    [filter
      [JwtRequestFilter.java, file]
    ]
  \end{forest}
  \caption{Structure of filter folder.}
\end{figure}


\subsubsection{\texttt{model}}

Models are mainly used to transmit data between client and server sides. There were two reasons. The first one is that client does not know some information, such as the creation date. The second one is that the client should not and not need to see some information, such as the user's password or the object's update date. Therefore, I used DTOs for any information transfer between the client and server sides.

In addition to DTOs, one class, called \texttt{MyUserDetails}, is not used for information transfer. It was needed for the authentication system. It helps by indicating how the necessary information is extracted from my \texttt{User} entity.

\begin{figure}[ht]
  \label{back-end-model-tree}
  \centering
  \begin{forest}
    pic dir tree,
    where level=0{}{% folder icons by default; override using file for file icons
      directory,
    },
    [model
      [AuthDTO.java, file]
      [AuthResponseDTO.java, file]
      [BookDTO.java, file]
      [BookResponseDTO.java, file]
      [BookUpdateDTO.java, file]
      [JwtRequest.java, file]
      [JwtResponse.java, file]
      [MyUserDetails.java, file]
      [RoleResponseDTO.java, file]
      [UserDTO.java, file]
      [UserResponseDTO.java, file]
      [UserUpdateDTO.java, file]
    ]
  \end{forest}
  \caption{Structure of model folder.}
\end{figure}


\subsubsection{\texttt{repo}}

Repositories are used to access the databases, and they make use of the \texttt{Spring JPA} by extending \texttt{JpaRepository}. Some essential operation, such as directly adding or finding by id, is inside the \texttt{JpaRepository} but if I want to add something special, I write the function name by using keywords, such as \texttt{findBy}, or \texttt{All}, and \texttt{Spring} generates the necessary queries in the background. This is so easy to use and increases the speed of development.

\begin{figure}[ht]
  \label{back-end-repo-tree}
  \centering
  \begin{forest}
    pic dir tree,
    where level=0{}{% folder icons by default; override using file for file icons
      directory,
    },
    [repo
      [BookRepository.java, file]
      [RoleRepository.java, file]
      [UserRepository.java, file]
    ]
  \end{forest}
  \caption{Structure of repo folder.}
\end{figure}

There is one for each main entity. \texttt{BookRepository} is used for operations on books, \texttt{UserRepository} is used for operations on users, and \texttt{RoleRepository} is used for operations on roles.


\subsubsection{\texttt{service}}

Services are the business layer of my app. All the operations, such as adding or removing the books or users, and logic are done in this layer. Controllers use the services for related operations; therefore, it can be said that there is one service for each controller.

Inside services, other services or necessary repositories are used. Accessing the database is done inside services with the help of repositories. Repositories contain several different implementations of the same operation for different needs. For example, several search functions exist for pageable and list data.

\begin{figure}[ht]
  \label{back-end-service-tree}
  \centering
  \begin{forest}
    pic dir tree,
    where level=0{}{% folder icons by default; override using file for file icons
      directory,
    },
    [service
      [BookListService.java, file]
      [BookService.java, file]
      [JwtUserDetailsService.java, file]
      [UserService.java, file]
    ]
  \end{forest}
  \caption{Structure of service folder.}
\end{figure}


\subsubsection{\texttt{util}}

This folder contains only one class, called \texttt{JwtTokenUtil}. This class contains several functions applicable to the \texttt{JWT} token, such as getting the username or expiration date from the token. Also, it can generate a new \texttt{JWT} token. \texttt{JWT} filter and \texttt{AuthController} excessively used this class.

\begin{figure}[ht]
  \label{back-end-util-tree}
  \centering
  \begin{forest}
    pic dir tree,
    where level=0{}{% folder icons by default; override using file for file icons
      directory,
    },
    [util
      [JwtTokenUtil.java, file]
    ]
  \end{forest}
  \caption{Structure of controller folder.}
\end{figure}
