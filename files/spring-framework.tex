\section{\texttt{Spring} Framework}

In web development, our next step was a framework. In the internship program, \texttt{Spring}, \href{https://spring.io/why-spring}{the world's most popular \texttt{Java} framework}, is chosen for that purpose. \texttt{Spring} is an open-source framework developed for \texttt{Java}. Generally, it is used for developing web apps with the \texttt{Java} Enterprise platform. For ease of reading, I will dive into the subsections and briefly explain what I learned about \texttt{Spring} and what they are.

\subsection{Introduction to \texttt{Spring}}

By using \texttt{Spring initializr}, we created our first spring project. While creating a project, our mentor addressed the features of \texttt{Spring initializr} and the dependencies part of it. While discussing the dependencies, we found out a problem with versions; therefore, our mentor talked about the general versioning methods.

\subsection{Controllers}

We started with the controllers. Controllers are part of the Spring Web model-view-controller (MVC) framework and meet the requests. By using the mapping annotations, requests are mapped to related functions inside the controller class. The functions usually return a response after necessary processes are done according to the request. In our projects, we usually used ``ResponseEntity'' classes because we returned an entity.

\subsection{Filter and Interceptor}

We continued with the filters. The filters are objects used to intercept the HTTP requests and responses of the application. Basically, they are the layer before sending the request to the controller and before sending a response to the client. Some common usages are logging requests and responses, logging request processing time, formatting request body or header, verifying authentication tokens, or compressing responses.

Interceptors are pretty similar to filters, but they act in \texttt{Spring Context}, so they are powerful enough to manage HTTP Request and Response but can implement more sophisticated behavior because they can access all \texttt{Spring} contexts. In other words, we can not use filters outside the web context, while \texttt{Spring} interceptors can be used anywhere because they are defined in the application context.

\subsection{Exception}

Then we dived into the exceptions. Firstly, we examined the default answer of \texttt{Spring} for errors and it can be said that default answer constitutes a security vulnerability because of the information the error provides. For example, someone can tell that this system uses the \texttt{Spring} framework by looking the default answer. We discussed how we could handle errors and change the behaviors and messages when an error occurs.

\subsection{Model, Entity, and Database}

% Model
After the error handling, we learned the Data Transfer Objects (DTOs), which are used to encapsulate data and carry this data between processes and their usages. I extensively used models in the final project to transfer data between processes and layers.

% Entity and Database
Since we will have been using the database, we had to make the database connection. \texttt{Spring} provides database connection capability with \texttt{Hibernate}, which is integrated inside \texttt{Spring} and makes the database connection and necessary database operation. 

\texttt{Hibernate} can also create the tables in the database by using entities provided by entity classes. Therefore, we used entity classes to tell the \texttt{Hibernate} to create the table in the database.

\subsection{Services, Repositories, and Configuration}

% Services and Repositories
Services are generally used for business logic, such as creating, storing, or changing data. I extensively used the services in my final project.

Beside, Repositories are the mechanism for encapsulating storage, retrieval, and search behavior that emulates a collection of objects. They are used for the access layer for accessing and making necessary operations in the database. We chose to use \texttt{Spring Data JPA} for this purpose. After making the required settings, \texttt{Spring} automatically connects itself to the database and gets ready to execute queries with little effort. In repositories, there is almost no need to write SQL queries executed in the database, even if it is allowed. By using the specific combination of some keys in the function names inside the repository interface, we provide necessary information to \texttt{Spring} so that \texttt{Spring} can create the proper queries and execute them by connecting to the database.

% Various Configuration
After the basics, we dived into the configuration of our apps, such as security, password encryption, or data loading when it is started.
