\section{Conclusion}

This summer practice was my first practice and the first experience in \texttt{Java} web development. The first part of my internship consisted of knowledge sharing. That part taught me so much about \texttt{Java}, Web Development, and \texttt{Spring}. This knowledge sharing was quite intense and sometimes hard to follow, and the last part of it was not beneficial for me due to the intensity. I had hard times about \texttt{React} and I had to spend time and efforts to learn and straighten the information that I misunderstood. I believe information sharing part may be planned better by spreading the program for at least one week more, five weeks at total.

The second part of the practice consisted of developing a project from scratch. The project part was pretty beneficial because before this project, I didn't have to set up the general structure for my projects and assignments at school. I was trying to do something that needed to be done, like implementing a function or data structure class, and was trying to fix bugs if they came up. Therefore, although I did not have much difficulty in my experience until this project, I can say that I had a lot of difficulty in this project, especially in the design part. The probable reason for this is that I have not been involved in a project of this scale before.

Although the requirements were given in the project, we were asked to think and develop many things such as thinking the design of the general flow. Until the end of the fourth week (last week of the normal internship program), we were asked to finish this project. Due to my limited time, I did not have time to think much, so I started developing it after a short thought process. This short-term thinking and inexperience made my job very difficult. I couldn't complete the project in the expected time due to the long time I wasted on unfocused parts such as design.

In short, I tried to do everything in the best way and move on to the next stage, instead of revealing something tangible on a part and then continuing to develop on it. But I realized too late that this wouldn't work in this type of development. Therefore, I was not able to finish my project at the end of the fourth week. However, I managed to finish my project with decent planning and developing until the end of the my internship.

In conclusion, in addition to many things I learned about \texttt{Java} and web development, I learned how not to develop a project from scratch and possible mistakes in a four-week period. In the last two weeks of my internship, I understood how a project should be developed and what needs to be done before it starts. Additionally, I do think I would choose to continue on web development even though I had hard time at some points.
