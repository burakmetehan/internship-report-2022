\section{Introduction to Web Development}

The first week of my internship was about the \texttt{Java} Core, whereas we focused on the enterprise version of \texttt{Java}, \texttt{Java EE}, or \texttt{Jakarta EE}, and web development in the second week.

\subsection{Introduction to \texttt{Java EE} Platform}

After learning the \texttt{Java} versions and their primary concern, we discussed enterprise-level applications and their needs. Then, we focused on how \texttt{Java} values developers, vendors, and businesses. Our mentor mentioned that the specifications are determined and defined concurringly, and vendors only compete at their implementation level in the \texttt{Java} world so that developers can use any \texttt{J2EE} implementation for development and deployment.

Additionally, I learned about layers that are the foundation of software architecture (presentation, application/business, data, and service layers) and types of software architectures (one-tier, two-tier, three-tier, and N-tier architectures) and their advantages and disadvantages.

\subsection{Introduction to Web Development on \texttt{Java} Platform}

After learning what websites and web applications are and their differences, we discussed web servers and installed one of the preferred web servers on \texttt{Java} World: `\texttt{Tomcat}'. Then, we focused on the basics, such as HTTP and WWW, Request and Response, DNS, of the web world.

After these basics, we learned the `\texttt{Servlets}', the basic concept and tool for Web development on \texttt{Java} world and \texttt{JSP}, which enables mixing static HTML content with unique code that produces the dynamic content.

\subsection{Developing \texttt{Java} Web Applications}

We learned the basic anatomy of a web application and web module structure. I believe these topics are pretty beneficial because knowing the web module's structure is essential in portability's development and deployment process. After that, we dived into HTTP and I was mainly surprised when I learned the greatness of the contents carried by a request and response. Then, we focused on Servlets which are the main fragment of \texttt{Java} web development. 

Additionally, we discussed the `\textit{Threading Issues}' caused by the multi-threaded approach, which is an approach to solving the problem of how one servlet object serves many clients. In this approach, instead of creating multiple things, the servlet container creates a separate thread for each invocation of \texttt{service()} method, and that thread runs all servlet methods.

We also discussed information-sharing techniques among servlets (ServletContext object, HttpSession object, Request attributes). Before learning these techniques, I used incredible nonsensical methods such as different web pages to carry information. However, these techniques are beneficial and make it easier to carry information between servlets.

\subsection{More on \texttt{Java} Web Applications}

We learned what a developer should do for an exception and error management by configuring a web descriptor, namely \texttt{web.xml}. This week, we generally used XML files for configurations; however, following weeks, we used annotation-based configurations.

We focused on session management and filters on the week's last days. We learned session tracking techniques in detail, which of them are tracking via IP address, user authentication, hidden form fields, URL rewriting, and cookies. In the filters part, we learned how filters actually work with other parts of the server. Then, we practiced how requests are used for authentication and security purposes.

Until this week, my knowledge about web development was pretty limited, and I was a novice to the web at this point. Therefore, these topics were confusing but enthusiastic for me.